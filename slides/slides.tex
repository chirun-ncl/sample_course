\documentclass[serif,10pt,compress]{beamer}
\usetheme{default}
\usecolortheme{default}
\usepackage{amsbsy,amsmath,latexsym,amsfonts,subfigure,enumerate}
\usepackage{hyperref}
\usepackage{graphicx}
\setbeamertemplate{navigation symbols}{}
\title{MAS3901}
\author{Phil Ansell}
\institute{MAS3901@ncl.ac.uk}
\date{}
\definecolor{dgreen}{rgb}{0.,0.4,0.}
\newcommand{\bqed}{\blacksquare}
\newcommand{\fblu}{\setbeamercolor{frametitle}{bg=white,fg=blue}}
\newcommand{\fred}{\setbeamercolor{frametitle}{bg=white,fg=blue}}
\newcommand{\fgreen}{\setbeamercolor{frametitle}{bg=white,fg=blue}}
\newcommand{\fblack}{\setbeamercolor{frametitle}{bg=white,fg=blue}}
\newcommand{\tw}{\textcolor{blue}}
\setbeamertemplate{blocks}[framed]
%\pgfpagesuselayout{2 on 1}[a4paper]
\newcommand{\aframetitle}[1]{\frametitle{\thepart.\thesection ~#1}}
\newcommand{\bframetitle}[1]{\frametitle{\thepart.\thesection.\thesubsection ~#1}}
\newcommand{\xvec}{\underline{x}}
\newcommand{\yvec}{\underline{y}}
\newcommand{\Xvec}{\underline{X}}
\newcommand{\Yvec}{\underline{Y}}
\newcommand{\thetavec}{\underline{\theta}}
\newcommand{\xvece}{\underline{x}=(x_1,x_2,\ldots,x_n)}
\newcommand{\Xrow}{X_1,X_2,\ldots,X_n}
\newcommand{\xrow}{x_1,x_2,\ldots,x_n}
\newcommand{\arow}{a_1,a_2,\ldots,a_n}
\newcommand{\ttheta}{\psfrag{t}{$\theta$}}
\newcommand{\asim}{\sim\hspace*{-9.6pt}\raisebox{0.5pt}{:}\hspace*{6pt}}
\newcommand{\aasim}{\sim\hspace*{-17pt}\raisebox{0.5pt}{:}\hspace*{6pt}}
\newcommand{\bigfrac}[2]{\frac{\displaystyle{#1}}{\displaystyle{#2}}}
\newcommand{\QED}{{\it Q.E.D.}}
\newcommand{\dsp}{\doublespace}
\newcommand{\ssp}{\singlespace}
\newcommand{\osp}{\onehalfspace}
\newcommand{\pencil}{\PencilRightDown}
\newcommand{\ffrac}[2]{\displaystyle{\genfrac{}{}{0pt}{}{#1}{#2}}}
\renewcommand{\vec}[1]{\boldsymbol{#1}}

% The following LaTeX code is required for processing to the web.
% Removing it may break the build process.
\newif\ifplastex\plastexfalse
\ifplastex
    \usepackage{embed}
    \usepackage{beamer}
    \newcommand{\cellcolor}[2][white]{}
\else
    \usepackage{comment}
    \usepackage{colortbl}
    \newcommand{\externaldocument}[1]{}
	\renewcommand\qed{\begin{flushright}$\blacksquare$\end{flushright}}
    \excludecomment{HTML}
    \newcommand{\numbas}[2][Visit the URL below to try a numbas exam:]{\bigskip\noindent\textbf{Test Yourself} \textrm{#1}\\\expandafter\url{#2}\par}
    \newcommand{\vimeo}[2][Visit the URL below to view a video:]{\bigskip\noindent\textbf{Video} \textrm{#1}\\\expandafter\url{https://player.vimeo.com/video/#2}\par}
	\newcommand{\youtube}[2][Visit the URL below to view a video:]{\bigskip\noindent\textbf{Video} \textrm{#1}\\\expandafter\url{https://www.youtube.com/embed/#2}\par}
\fi

\begin{document}

\begin{frame}
\begin{center}
\Huge
\textbf{MAS3901} \\
\textbf{Applied Probability}\\
\vspace{0.1cm}
\huge
\textbf{Lecturer: Dr Phil Ansell}
\vspace{0.2cm}

Office: \textsf{3.05, Herschel Building} \\ Phone: \textsf{0191 208 6344} \\ \ Email: \texttt{MAS3901@ncl.ac.uk} 
\vspace{0.2cm}
\end{center}
\end{frame}

\begin{frame}[t]
	\frametitle{Module Timetable}
	\begin{center}
		\begin{tabular}{|l||c|c|c|c|}
			\hline
			Week & Mon 12-1 & Tue 9-10 & Wed 12-1 & Thu 4-6 \\
			\hline
			1 (2/10) &  & \cellcolor[gray]{0.5} Lecture 1 & \cellcolor[gray]{0.5} Lecture 2 &  \\
			\hline
			2 (9/10) & \cellcolor[gray]{0.5} Lecture 3 & \cellcolor[gray]{0.5} Lecture 4 & \cellcolor[gray]{0.5} Pr Class 1 & \\
			\hline
			3 (16/10) & & \cellcolor[gray]{0.5} Lecture 5 & \cellcolor[gray]{0.5} Lecture 6& \\
			\hline
			4 (23/10) & \cellcolor[gray]{0.5} Lecture 7 & \cellcolor[gray]{0.5} Lecture 8 & \cellcolor[gray]{0.5} Pr Class 2 &  \\
			\hline
			5 (30/10) & \cellcolor[gray]{0.5} Test 1 & & & \cellcolor[gray]{0.8} Lecture 9 \\
			\hline
			6 (6/11) & \cellcolor[gray]{0.65} Lecture 10 & \cellcolor[gray]{0.65} Drop-in & \cellcolor[gray]{0.65} Lecture 11  & \cellcolor[gray]{0.8} Comp 1 \\
			\hline
			7 (13/11) & \cellcolor[gray]{0.65} Lecture 12 &  & \cellcolor[gray]{0.65} Lecture 13 & \cellcolor[gray]{0.8} Deadline \\
			\hline
			8 (20/11) & \cellcolor[gray]{0.65} Lecture 14 & \cellcolor[gray]{0.65} Drop-in & \cellcolor[gray]{0.65} Lecture 15 &  \\
			\hline
			9 (27/11) & \cellcolor[gray]{0.65} Lecture 16 & \cellcolor[gray]{0.65} Drop-in & \cellcolor[gray]{0.65} Pr Class 3 &  \\
			\hline
			10 (4/12) & &\cellcolor[gray]{0.65} Test 2 & &  \\
			\hline
			11 (11/12) & \cellcolor[gray]{0.5} Lecture 17 & & \cellcolor[gray]{0.5} Pr Class 4 & \\
			\hline
			12 (8/1) & Revision 1 & & Revision 2 & \\
			\hline
		\end{tabular}
	\end{center}
	Mon 12-1: Herb LT2; \\
	Tue 9-10: BSTC 1.46; \\
	Wed 12-1: Herb LT3; \\
	Thu 4-6: Herschel Learning Lab.
\end{frame}
\begin{frame}[t]
	\frametitle{Introduction}
	\begin{enumerate}[--]
		\item Coursework will make up {\bf 10\%} of your final mark: {\bf Test 1 (L1-8, 2.5\%)}, {\bf Group Project (5\%)}, {\bf Test 2 (L10-16, 2.5\%)}. 
		\item Since there are no assessed homeworks, in weeks {\bf 1-4}, {\bf 6-9} and {\bf 11}, I will provide you with a set of problems (and later, solutions) for you to practice on. Further questions will be discussed in the Problem Classes.
		\item Announcements will be via email and posted on {\bf Blackboard}. You should check your email regularly!
		\item Please send any email requests to the module email account: {\bf MAS3901@ncl.ac.uk} and this will be checked between {\bf 8am} and {\bf 9am} each morning and whenever possible {\bf but please do not expect an immediate response to your email}. 
		\item I operate an {\bf open door} policy. 
		\item The group project must be handed in by the given deadline and will count as zero for assessment purposes unless a good reason is given via a PEC. {\bf This module does not allow extensions.}
		\item Note that you should be familiar with Degree Programme Handbook and in particular {\bf Section 7: Expectations}. 
		\item All the lectures for this module will be recorded by {\bf ReCap}. 
		
	\end{enumerate}
\end{frame}

\begin{frame}[t]
	\frametitle{Series Results}
	\begin{eqnarray*}
		\sum_{x=0}^{\infty} q^x &=& \frac{1}{1-q}, \ |q|<1. \\
		\frac{d}{dx}\left(\sum_{x=0}^{\infty} q^x \right) &=& \sum_{x=0}^{\infty} xq^{x-1}=\sum_{x=1}^{\infty} xq^{x-1} = \frac{1}{(1-q)^2}, \ |q|<1. \\
		\sum_{x=0}^{N} q^x &=& \frac{1-q^{N+1}}{1-q}. \\
		\sum_{n=0}^{\infty} \frac{x^n}{n!} &=& 1+x+\frac{x^2}{2!}+\frac{x^3}{3!}+\ldots = e^x
	\end{eqnarray*}
\end{frame}

\begin{frame}[t]
	\frametitle{Expectation}
{\bf Discrete Random Variables}
\begin{eqnarray*}
E[g(X)] &=& \sum_{x \in S} g(x) Pr(X=x) \\
Var(X)  &=& E[X^2] - E[X]^2.
\end{eqnarray*}
{\bf Continuous Random Variables}
\begin{eqnarray*}
E[g(X)] &=&  \int_{-\infty}^{\infty} g(x) f_X(x) dx \\
Var(X)  &=& E[X^2]-E[X]^2.
\end{eqnarray*}
\end{frame}

\begin{frame}[t]
\frametitle{Families of Discrete Random Variables I}
{\bf Bernoulli random variables, $X \sim Bern(p)$}
\begin{eqnarray*}
Pr(X=x) &=& \left\{ \begin{array}{cc} p, & x=1, \\ 1-p, & x=0. \end{array} \right.
\end{eqnarray*}
\[
E[X] = p \qquad Var(X) = p(1-p).
\]
{\bf Binomial random variables, $X \sim Bin(n,p)$} 
\begin{eqnarray*}
Pr(X=x) &=& \left( \begin{array}{c} n \\ x
      \end{array} \right) p^x (1-p)^{n-x},\  x=0,1,\ldots,n,
\end{eqnarray*}
\[
E[X] = np \qquad Var(X) = np(1-p).
\]
{\bf Poisson random variables, $X \sim Po(\lambda), \ \lambda>0$}
\begin{eqnarray*}
Pr(X=x) &=& \frac{\lambda^x e^{-\lambda}}{x!}, \ x=0,1,2,\ldots.
\end{eqnarray*}
\[
E[X] = \lambda \qquad Var(X) = \lambda.
\]
\end{frame}

\begin{frame}[t]
\frametitle{Families of Discrete Random Variables II}

{\bf Geometric random variables, $X \sim Geom(p)$}
\begin{eqnarray*}
Pr(X=x) &=& p(1-p)^{x-1}, \ x=1,2,\ldots.
\end{eqnarray*}
\[
E[X] = \frac{1}{p} \qquad Var(X) = \frac{1-p}{p^2}.
\]
{\bf Negative Binomial random variables, $X \sim NBin(r,p)$} 
\begin{eqnarray*}
Pr(X=x) &=& \left( \begin{array}{cc} x-1 \\ r-1 \end{array} \right) p^r (1-p)^{n-r}, \ x=r,r+1,\ldots.
\end{eqnarray*}
\[
E[X] = \frac{r}{p} \qquad Var(X) = \frac{r(1-p)}{p^2}. 
\]
{\bf Discrete Uniform random variables, $X \sim U(1,2,\ldots,n)$}
\begin{eqnarray*}
Pr(X=x) &=& \frac{1}{n}, \ x=1,2,\ldots,n.
\end{eqnarray*}
\[
E[X] = \frac{n+1}{2} \qquad Var(X) = \frac{n^2-1}{12}.
\]
\end{frame}


\begin{frame}[t]
\frametitle{Families of Continuous Random Variables I}

{\bf Uniform random variables}, $X \sim U(a,b)$,
\begin{eqnarray*}
f_X(x) &=& \frac{1}{b-a}, \ a < x < b.
\end{eqnarray*}
\[
E[X]=\frac{a+b}{2} \qquad Var(X) = \frac{(b-a)^2}{12}.
\]
{\bf Exponential random variables}, $X \sim Exp(\lambda)$, 
\begin{eqnarray*}
f_X(x)&=& \lambda e^{-\lambda x}, \ x>0, \ \lambda>0.
\end{eqnarray*}
\[
E[X]= \frac{1}{\lambda} \qquad Var(X) = \frac{1}{\lambda^2}.
\]
{\bf Normal random variables}, $X \sim N(\mu,\sigma^2)$, 
\[
f_X(x) = \frac{1}{\sqrt{2\pi \sigma^2}} e^{-\frac{(x-\mu)^2}{2\sigma^2}}, \ -\infty < x < \infty, \ -\infty < \mu < \infty, \ \sigma>0.
\]
\[
E[X]= \mu \qquad Var(X) = \sigma^2.
\]
\end{frame}

\begin{frame}[t]
\frametitle{Families of Continuous Random Variables II}
{\bf Gamma random variables}, $X \sim Ga(n,\lambda)$,
\[
f_X(x) = \frac{\lambda^{n}}{\Gamma(n)} x^{n-1}
e^{-\lambda x}, \qquad x>0, \ n>0,\ \lambda>0, \ \Gamma(n)=(n-1)!
\]
\[
E[X] = \frac{n}{\lambda} \qquad Var(X) = \frac{n}{\lambda^2}. 
\]
{\bf Beta random variables}, $X \sim Beta(a,b)$,
\begin{eqnarray*}
f_X(x) &=& \frac{1}{B(a,b)} x^{a-1}(1-x)^{b-1} \\
&=&  \frac{\Gamma(a+b)}{\Gamma(a)\Gamma(b)} x^{a-1}(1-x)^{b-1},\qquad 0<x<1, \ a>0, \ b >0. 
\end{eqnarray*}
\[
E[X]=\frac{a}{a+b} \qquad  Var(X) = \frac{ab}{(a+b)^2(a+b+1)}.
\]
\end{frame}
\end{document}
